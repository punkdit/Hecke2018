
%\documentclass[11pt,oneside,twocolumn]{article} 
\documentclass[11pt,oneside]{article} 

\usepackage{a4wide}

\usepackage{amsmath}
\usepackage{color}
%\usepackage{natbib} % kills arxiv 
\usepackage{framed}
%\usepackage{cite}
\usepackage{tikz}
\usepackage{tikz-cd}

\RequirePackage{amsmath}
\RequirePackage{amssymb}
\RequirePackage{amsthm}
%\RequirePackage{algorithmic}
%\RequirePackage{algorithm}
%\RequirePackage{theorem}
%\RequirePackage{eucal}
\RequirePackage{color}
\RequirePackage{url}
\RequirePackage{mdwlist}

\RequirePackage{rotating}


\RequirePackage[all]{xy}
%\_CompileMatrices
%\RequirePackage{hyperref}
\RequirePackage{graphicx}
\RequirePackage{subcaption}
%\RequirePackage[dvips]{geometry}

\usepackage{xcolor}
\usepackage{amsmath,amsfonts,amssymb}
\usepackage{graphicx}
%\usepackage[caption=false]{subfig}
\usepackage{enumerate}
\usepackage{mathrsfs}
\usepackage{mathtools}




% -------------- _Commands ----------------------

\makeatletter
\newcommand{\verbatimfont}[1]{\renewcommand{\verbatim@font}{\ttfamily#1}}
\makeatother

\newcommand{\Eref}[1]{(\ref{#1})}
\newcommand{\Fref}[1]{Fig.~\ref{#1}}
%\newcommand{\Aref}[1]{Appendix~\ref{#1}}
\newcommand{\SRef}[1]{Section~\ref{#1}}

\newcommand{\todo}[1]{\ \textcolor{red}{\{#1\}}\ }

\newcommand{\Aut}{\mathrm{Aut}}
\newcommand{\Stab}{\mathrm{Stab}}
\newcommand{\Fix}{\mathrm{Fix}}
\newcommand{\Orbit}{\mathrm{Orbit}}
\newcommand{\Dim}{\mathrm{Dim}}
\newcommand{\Tr}{\mathrm{Tr}}
\newcommand{\Det}{\mathrm{Det}}

\newcommand{\Ker}{\mathrm{ker}}
\newcommand{\Kernel}{\mathrm{ker}}
\newcommand{\Image}{\mathrm{im}}
\newcommand{\Coker}{\mathrm{coker}}
\newcommand{\Cokernel}{\mathrm{coker}}
\newcommand{\Coimage}{\mathrm{coim}}

\newcommand{\GL}{\mathrm{GL}}
\newcommand{\AGL}{\mathrm{AGL}}
\newcommand{\Unitary}{\mathrm{U}}

\newcommand{\Complex}{\mathbb{C}}
\newcommand{\CC}{\mathbb{C}}
\newcommand{\Natural}{\mathbb{N}}
\newcommand{\NN}{\mathbb{N}}
\newcommand{\Integer}{\mathbb{Z}}
\newcommand{\ZZ}{\mathbb{Z}}

\newcommand{\Field}{\mathbb{F}}
\newcommand{\Code}{\mathcal{C}}
\newcommand{\End}{\mathrm{End}}
\newcommand{\Mon}{{\mathcal{M}^\otimes}}
\newcommand{\Hom}{\mathrm{Hom}}
\newcommand{\Inv}{\mathrm{Inv}}
\newcommand{\Span}{\mathrm{Span}}
\newcommand{\Hilb}{\mathcal{H}}
\newcommand{\Rey}{\mathcal{R}}
\newcommand{\tensor}{\otimes}

\usepackage{mathtools}
\DeclarePairedDelimiter\ceil{\lceil}{\rceil}
\DeclarePairedDelimiter\floor{\lfloor}{\rfloor}


% Lemma, proof, theorem, etc.
\newcommand\nounderline[1]{ #1} 
\newcommand\dolemma[1]{\vskip 5pt \noindent{\bf \underline{Lemma #1.}\ }}
\newcommand\doproposition[1]{\vskip 5pt \noindent {\bf \underline{Proposition #1.}\ }}
\newcommand\dotheorem[1]{\vskip 5pt \noindent {\bf \underline{Theorem #1.}\ }}
\newcommand\doconjecture[1]{\vskip 5pt \noindent {\bf \underline{Conjecture #1.}\ }}
\newcommand\dodefn[1]{\vskip 5pt \noindent {\bf \underline{Definition #1.}\ }}
\newcommand\docorollary[1]{\vskip 5pt \noindent {\bf \underline{Corollary #1.}\ }}
\newcommand\doexample[1]{\vskip 5pt \noindent {\bf \underline{Example #1.}\ }}
\newcommand\doexamples[1]{\vskip 5pt \noindent {\bf \underline{Examples #1.}\ }}
\newcommand\doproof{\vskip 5pt \noindent{\bf \nounderline{Proof:}\ }}
\newcommand\doremark[1]{\vskip 5pt \noindent{\bf \underline{Remark #1.}\ }}

%\newcommand\tombstone{\rule{.36em}{2ex}\vskip 5pt}
%\newcommand\tombstone{\qedsymbol\vskip 5pt}
\newcommand\tombstone{\rule{.6em}{.6em}}

\newcounter{numitem}
\newcommand{\numitem}[1]{\refstepcounter{numitem}\thenumitem\label{#1}}

% braket notation...
\newcommand{\ket}[1]{|{#1}\rangle}
\newcommand{\expect}[1]{\langle{#1}\rangle}
\newcommand{\bra}[1]{\langle{#1}|}
\newcommand{\ketbra}[2]{\ket{#1}\!\bra{#2}}
\newcommand{\braket}[2]{\langle{#1}|{#2}\rangle}

% Categories
\newcommand{\Set}{\mathbf{Set}}
\newcommand{\FinSet}{\mathbf{FinSet}}
\newcommand{\GSet}{\mathbf{GSet}}
\newcommand{\GRep}{\mathbf{GRep}}

\newcommand{\thinplus}{\!+\!}

\renewcommand{\arraystretch}{1.2}


\title{Notes}

\author{James Dolan}


\date{\today}

\flushbottom

\begin{document}

\maketitle

% ----------------------------------------------------------------------------
%
%

\section{Some matrixes of formal power serieses}


It's well-known that for a commutative ring $A$, the category of free
$A$-modules is equivalent to the category where an object is a set and a
morphism is a column-finite matrix with entries in $A$, the composition
of morphisms being accomplished by the usual procedure for matrix
multiplication.

In the case where $A$ is the commutative ring of formal power with
coefficients in a field $k$, however, there's an interesting relaxation
of the column-finiteness restriction that still allows us to form a
category.  I'll describe this relaxation, and then begin to try to
explain what's the use of the resulting category.

Definition: A matrix $m$ of formal power serieses (in one variable $x$) is
defined to be ``degree-wise column-finite" iff for any natural number
$d$, $m$ becomes column-finite after truncating the power serieses to
degree $d$.

The naive rule for matrix multiplication still works on these matrixes.

The main use of the resulting category is that it's the
Kleisli-category of a (non-finitary) commutative monad $A$ on the
category of sets.  (The Kleisli-category of a monad is the category of
it's free algebras; the monad can then be recovered as the monad of
the adjunction between its base-category and it's Kleisli-category.)

Durov's thesis \cite{Durov2007} defines a ``generalized commutative ring" to be a
\_finitary\_ commutative monad on the category of sets, and defines the
(not-necessarily free) ``modules" of a generalized commutative ring to
be the (not-necessarily free) algebras of the corresponding monad.
The finitariness restriction is not essential, however and so we can
regard the commutative monad $A$ as a ``generalized generalized
commutative ring" (or ``generalized commutative ring" for short).

We can then say a couple of interesting things about the modules of
our generalized commutative ring $A$:

1.  An $A$-module is essentially the same thing as a vector space $V$
equipped with a linear operator $L$ satisfying Property $P$ described in
manxometove's reddit post on ``Property dual to local nilpotence" [2].

2.  The category of $A$-modules is the syntactic category of the
algebro-geometric theory of ``a point of the co-commutative co-algebra
of polynomials in $x$ with the 
co-multiplication given by $x \mapsto x\tensor 1 + 1\tensor x$'',
where a "point" of a
co-commutative co-algebra is a morphism from the terminal
co-commutative co-algebra.  In other words the $A$-modules are the
"believers" in a morphism of $k$-co-algebras from the base field $k$ to
the co-algebra $k[x]$ as described above.

(Many details need to be checked here, and of course the conceptual
picture needs to be clarified much more, but I'll stop here for now.)

\subsection{Addendum}

Perhaps another way to describe (some of) what's going on here is this:

The truncations of the ring of polynomials in $x$ obtained by modding
out by $x^d$ form an inverse system of commutative rings.  If you take
the inverse limit of this inverse system in the category of
commutative rings, or even in the category of Durov's generalized
commutative rings, then you get the commutative ring of formal power
serieses.  However if you take the inverse limit in the category of
"generalized generalized commutative rings" (that is, removing Durov's
requirement that the commutative monads on the category of sets have
to be finitary), then you get something potentially more interesting,
namely the generalized generalized commutative ring ``$A$'' described in
``Some matrixes of formal power serieses".

Meanwhile, I didn't get around yet to describing the duality between
the $A$-modules (that is, the believers in a point of the co-algebra
with co-multiplication $x \mapsto x\tensor 1 + 1\tensor x$) and the co-modules of that
co-algebra ....



\bibliography{refs2}{}
\bibliographystyle{abbrv}


\end{document}



