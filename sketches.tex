
\documentclass[11pt,oneside]{article} 

\usepackage{a4wide}

\usepackage{amsmath}
\usepackage{color}
%\usepackage{natbib} % kills arxiv 
\usepackage{framed}
%\usepackage{cite}
\usepackage{tikz}
\usepackage{tikz-cd}

\RequirePackage{amsmath}
\RequirePackage{amssymb}
\RequirePackage{amsthm}

%\RequirePackage{algorithmic}
%\RequirePackage{algorithm}
%\RequirePackage{theorem}
%\RequirePackage{eucal}
\RequirePackage{color}
\RequirePackage{url}
\RequirePackage{mdwlist}

\RequirePackage{rotating}


\RequirePackage[all]{xy}
%\_CompileMatrices
%\RequirePackage{hyperref}
\RequirePackage{graphicx}
%\RequirePackage[dvips]{geometry}

\usepackage{xcolor}
%\usepackage{amsmath,amsfonts,amssymb}
\usepackage{graphicx}
\usepackage[caption=false]{subfig}
\usepackage{enumerate}
\usepackage{mathrsfs}
\usepackage{bbm}

% -------------- Commands ----------------------

\makeatletter
\newcommand{\verbatimfont}[1]{\renewcommand{\verbatim@font}{\ttfamily#1}}
\makeatother

\newcommand{\Eref}[1]{(\ref{#1})}
\newcommand{\Fref}[1]{Fig.~\ref{#1}}
%\newcommand{\Aref}[1]{Appendix~\ref{#1}}
\newcommand{\SRef}[1]{Section~\ref{#1}}

\newcommand{\todo}[1]{\ \textcolor{red}{#1}\ }

\newcommand{\Aut}{\mathrm{Aut}}
\newcommand{\Hom}{\mathrm{Hom}}
%\newcommand{\hom}{\mathrm{hom}} % internal hom ?
\newcommand{\Stab}{\mathrm{Stab}}
\newcommand{\Fix}{\mathrm{Fix}}
\newcommand{\Orbit}{\mathrm{Orbit}}
\newcommand{\Ker}{\mathrm{Ker}}
\newcommand{\Image}{\mathrm{Im}}
\newcommand{\Dim}{\mathrm{Dim}}
\newcommand{\Complex}{\mathbb{C}}
\newcommand{\Integer}{\mathbb{Z}}
\newcommand{\GL}{\mathrm{GL}}
\newcommand{\Field}{\mathbb{F}}

% Lemma, proof, theorem, etc.
\newcommand\nounderline[1]{ #1} 
\newcommand\dodef[1]{\vskip 5pt \noindent{\bf \underline{Definition #1.}\ }}
\newcommand\dolemma[1]{\vskip 5pt \noindent{\bf \underline{Lemma #1.}\ }}
\newcommand\doproposition[1]{\vskip 5pt \noindent {\bf \underline{Proposition #1.}\ }}
\newcommand\dotheorem[1]{\vskip 5pt \noindent {\bf \underline{Theorem #1.}\ }}
\newcommand\docorollary[1]{\vskip 5pt \noindent {\bf \underline{Corollary #1.}\ }}
\newcommand\doexample[1]{\vskip 5pt \noindent {\bf \underline{Example #1.}\ }}
\newcommand\doproof{\vskip 5pt \noindent{\bf \nounderline{Proof:}\ }}

\newcommand\tombstone{\rule{.42em}{1.4ex}\vskip 5pt}

\newcounter{numitem}
\newcommand{\numlabel}[1]{\refstepcounter{numitem}\thenumitem\label{#1}}
\newcommand{\numitem}{\refstepcounter{numitem}\thenumitem}

% braket notation...
\newcommand{\ket}[1]{|{#1}\rangle}
\newcommand{\expect}[1]{\langle{#1}\rangle}
\newcommand{\bra}[1]{\langle{#1}|}
\newcommand{\ketbra}[2]{\ket{#1}\!\bra{#2}}
\newcommand{\braket}[2]{\langle{#1}|{#2}\rangle}

% Categories
\newcommand{\Set}{\mathbf{Set}}
\newcommand{\FinSet}{\mathbf{FinSet}}
\newcommand{\GSet}{\mathbf{GSet}}
\newcommand{\GRep}{\mathbf{GRep}}

\newcommand{\thinplus}{\!+\!}

\renewcommand{\arraystretch}{1.2}



\title{The Belief Method}


\date{\today}

\flushbottom

\begin{document}

\maketitle

%\begin{abstract}
%\end{abstract}

%\tableofcontents

%\doublespacing
%\onehalfspacing



\newcommand{\ACat}{\mathsf{A}}
\newcommand{\BCat}{\mathsf{B}}
\newcommand{\CCat}{\mathsf{C}}
\newcommand{\DCat}{\mathsf{D}}

\newcommand{\CatSet}{\mathsf{Set}}
\newcommand{\Ob}{\mathrm{Ob}}
\newcommand{\Mor}{\mathrm{Mor}}

\section*{Constructing locally presentable categories}

An alternate name for a locally presentable category is
a {\it smally sketched colimit theory} (\cite{Dolan} 2019-07-19).

\dodef{\numitem} (\cite{Dolan} 2019-07-22)
A {\it locally presentable category} is defined recursively as:

{\bf (Axiom)} The terminal category $\mathbbm{1}$ is locally presentable.
This is the \emph{theory of nothing}, the \emph{empty sketch}.

{\bf (0-cell)} 
Given a locally presentable category $\mathsf{A}$,
we freely adjoin an object to $\mathsf{A}$ giving
the locally presentable category 
$$
    \mathsf{B} = \mathsf{A}\times\mathsf{Set}.
$$
The new object in $\BCat$ is $(0,\star)\in \mathsf{A}\times\mathsf{Set}.$
This construction adds {\it stuff}.
\todo{We are adding an object to the sketch.}

{\bf (1-cell)} 
Given a locally presentable category $\mathsf{A}$,
and objects $c,d\in \mathsf{A}$,
we freely adjoin a morphism $f:d\to c$ to $\mathsf{A}$ giving
the locally presentable category 
$\mathsf{B}$ with
$$
    \Ob(\BCat) := \{ (b,g) : b\in\Ob(\ACat), g:\Hom(c,b)\to\Hom(d,b) \}
$$
and morphism $(b_1,f_1)\to(b_2, f_2)$ any $f:b_1\to b_2$ in $\ACat$ such that
$$
\begin{tikzcd}
 \Hom(c,b_1) \arrow{d}{\Hom(c,f)} \arrow{r}{f_1} &  \Hom(d,b_1) \arrow{d}{\Hom(d,f)} \\
 \Hom(c,b_2) \arrow{r}{f_2} &  \Hom(d,b_2) 
\end{tikzcd}
$$
commutes.
This construction adds {\it structure}.
\todo{We are adding a morphism to the sketch.}

{\bf (2-cell)}
Given a locally presentable category $\mathsf{A}$,
with morphism $f:d\to c$ we freely adjoin an inverse $g:c\to d$
giving the locally presentable category $\BCat$
as the full subcategory of $\ACat$
with objects 
$ b\in\Ob(\ACat) $ such that  $\Hom(f,b):\Hom(c,b)\to\Hom(d,b)$
        is invertible in $\CatSet$.
This construction adds a {\it property}.
\todo{We are adding a relation (formal inverse) to the sketch.}

We combine steps {\bf (0,1,2-cell)} a set amount of times 
\todo{ie. the steps are indexed by a set}.
\tombstone

The next theorem shows that we are doing a kind of 
\emph{categorified linear algebra}.

\dotheorem{\numitem}
(\cite{Riehl2017} 4.6.17)
A functor $F:\ACat\to\BCat$ between locally
presentable categories is a left adjoint
if and only if it is cocontinuous.
\tombstone

\section*{Belief doctrines}


\dodef{\numitem}
A \emph{belief doctrine} is a monad on the bicategory of
locally presentable categories.
\tombstone

Given a belief doctrine $\DCat$, 
we sometimes refer to its algebras as its
\emph{theories}, just as with Lawvere's definition of a doctrine, 
and to
homomorphisms between the algebras as \emph{interpretations} between the
theories.

\dodef{\numitem} 
Given a belief doctrine $\DCat$ 
and an interpretation $T \xrightarrow{J} U$
of $\DCat$-theories, a \emph{$J$-belief} on a 
$x\in\Ob(T)$
is equivalently either a
lift of $x$ along the right-adjoint of $J$, or a continuous extension of
$\Hom(-,x)$ along $T^{op} \xrightarrow{J^{op}} U^{op}$.
\tombstone

Example. ``Categorified symmetric algebra'' is a belief doctrine; its
theories are the symmetric-monoidal locally presentable categories,
and its interpretations are the symmetric-monoidal left-adjoint functors
between them.


%\section*{Sketches}
%
%\dodef{\numitem} (\cite{Adamek1994} 1.49)
%
%References: \cite{Adamek1994} \cite{Durov2007} \cite{Riehl2017} \cite{Heunen2019}


%\section*{Appendix}
%
%\subsection*{Commutative Algebra}
%
%\dodef{\numitem}
%A commutative ring is {\it local} when it has a unique
%maximal idea. \tombstone
%
%
%
%\subsection*{Adjoint functors}
%
%We wish to categorify linear algebra and also commutative algebra.
%
%A functor that preserves colimits is called {\it cocontinuous.}
%\dotheorem{\numitem}
%(\cite{Riehl2017} 4.5.3)
%Any left adjoint functor is cocontinuous.
%\tombstone
%
%\dodef{\numitem}
%(\cite{Riehl2017} 4.5.12)
%A {\it reflective subcategory}
%of a category $\ACat$ is a full subcategory $\BCat$
%where the inclusion
%$ \BCat \rightarrowtail \ACat $ 
%admits a left adjoint, called the {\it reflector}
%or {\it localization}:
%$$
%\begin{tikzcd}
%    \ACat \arrow{r}{L} & \BCat \arrow[d, rightarrowtail, "\dashv"'] &          \\
%                       & \ACat \arrow{r}{L}              & \BCat    
%\end{tikzcd}
%$$
%\tombstone


\bibliography{refs2}{}
\bibliographystyle{abbrv}

\end{document}
