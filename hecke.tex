
\documentclass[11pt,oneside]{article} 

\usepackage{a4wide}

\usepackage{amsmath}
\usepackage{color}
%\usepackage{natbib} % kills arxiv 
\usepackage{framed}
%\usepackage{cite}
\usepackage{tikz}
\usepackage{tikz-cd}

\RequirePackage{amsmath}
\RequirePackage{amssymb}
\RequirePackage{amsthm}
%\RequirePackage{algorithmic}
%\RequirePackage{algorithm}
%\RequirePackage{theorem}
%\RequirePackage{eucal}
\RequirePackage{color}
\RequirePackage{url}
\RequirePackage{mdwlist}

\RequirePackage[all]{xy}
%\_CompileMatrices
%\RequirePackage{hyperref}
\RequirePackage{graphicx}
%\RequirePackage[dvips]{geometry}

\usepackage{xcolor}
\usepackage{amsmath,amsfonts,amssymb}
\usepackage{graphicx}
\usepackage[caption=false]{subfig}
\usepackage{enumerate}
\usepackage{mathrsfs}

% -------------- Commands ----------------------

\newcommand{\Eref}[1]{(\ref{#1})}
\newcommand{\Fref}[1]{Fig.~\ref{#1}}
%\newcommand{\Aref}[1]{Appendix~\ref{#1}}
\newcommand{\SRef}[1]{Section~\ref{#1}}

\newcommand{\todo}[1]{\ \textcolor{red}{\{#1\}}\ }

\newcommand{\Aut}{\mathrm{Aut}}
\newcommand{\Hom}{\mathrm{Hom}}
\newcommand{\Stab}{\mathrm{Stab}}
\newcommand{\Fix}{\mathrm{Fix}}
\newcommand{\Orbit}{\mathrm{Orbit}}
\newcommand{\Ker}{\mathrm{Ker}}
\newcommand{\Image}{\mathrm{Im}}
\newcommand{\Dim}{\mathrm{Dim}}
\newcommand{\Complex}{\mathbb{C}}
\newcommand{\GL}{\mathrm{GL}}
\newcommand{\Field}{\mathbb{F}}

% Lemma, proof, theorem, etc.
\newcommand\nounderline[1]{ #1} 
\newcommand\dolemma[1]{\vskip 5pt \noindent{\bf \underline{Lemma #1.}\ }}
\newcommand\doproposition[1]{\vskip 5pt \noindent {\bf \underline{Proposition #1.}\ }}
\newcommand\dotheorem[1]{\vskip 5pt \noindent {\bf \underline{Theorem #1.}\ }}
\newcommand\docorollary[1]{\vskip 5pt \noindent {\bf \underline{Corollary #1.}\ }}
\newcommand\doexample[1]{\vskip 5pt \noindent {\bf \underline{Example #1.}\ }}
\newcommand\doproof{\vskip 5pt \noindent{\bf \nounderline{Proof:}\ }}

\newcommand\tombstone{\rule{.36em}{2ex}\vskip 5pt}

\newcounter{numitem}
\newcommand{\numitem}[1]{\refstepcounter{numitem}\thenumitem\label{#1}}

% braket notation...
\newcommand{\ket}[1]{|{#1}\rangle}
\newcommand{\expect}[1]{\langle{#1}\rangle}
\newcommand{\bra}[1]{\langle{#1}|}
\newcommand{\ketbra}[2]{\ket{#1}\!\bra{#2}}
\newcommand{\braket}[2]{\langle{#1}|{#2}\rangle}

% Categories
\newcommand{\Set}{\mathbf{Set}}
\newcommand{\FinSet}{\mathbf{FinSet}}
\newcommand{\GSet}{\mathbf{GSet}}
\newcommand{\GRep}{\mathbf{GRep}}



\title{Notes on Hecke Operators}

\author{Simon Burton}


\date{\today}

\flushbottom

\begin{document}

\maketitle

%\begin{abstract}
%\end{abstract}

%\tableofcontents

%\doublespacing
%\onehalfspacing

\section{Introduction}

Group representation theory concerns itself with
homomorphisms from a group $G$ to the general
linear group over a field $k:$
$$
    G \to \GL(n, k).
$$
The definition of the general linear group
makes sense not just for
a field $k$, but also for a ring, or even a 
semi-ring (a ring without additive inverses.)
In particular, we consider 
the semi-ring of truth-values:
$$
    \Field_1 = \{\mathrm{false}, \mathrm{true}\}
$$
with addition as disjunction and multiplication
as conjunction.
The notation $\Field_1$ refers to the ``field
with one element'',
which however is not a field and doesn't have one element \cite{Lorscheid2018}. 
We have the following
\dotheorem{\numitem{F1thm}}
The group $\GL(n, \Field_1)$ consists of $n\times n$
permutation matrices, and is therefore isomorphic
to the permutation group $S_n$.
\doproof
Adapted from \cite{Speyer2011}.
\tombstone

And so we find that group representation theory over the
semi-ring of truth values is the theory of groups acting on sets.

\section{Groups acting on sets}

We say a group $G$ \emph{acts} on a set $X$ when there
is a group homomorphism:
$$
    G \to \Aut(X).
$$
We choose not to name this homomorphism, and instead
confuse the elements of $G$ with their image in $\Aut(X).$
In this way we understood expressions such as $gx$ for $g\in G, x\in X.$
This is similar to how a field acts on a vector space:
we don't usually write the homomorphism, and instead just
let elements of the field act on the vectors (on the left.)

We also call this setup a $G$-set $X$.

%A map from a group action $G\to\Aut(X)$ to $G\to\Aut(Y)$
A map of $G$-sets $X\to Y$
is a set function $f:X\to Y$ that commutes with the group action.
That is, for every $g\in G$ we have the commuting square:
\[
\begin{tikzcd}
 X \arrow{d}{f} \arrow{r}{g} &  X \arrow{d}{f} \\
 Y \arrow{r}{g} &  Y 
\end{tikzcd}
\]
Thinking of $G$ as a one object category, a group action is
then a set-valued functor and we see that a map of $G$-sets
is the same as a natural transformation of functors.
This gives the category of $G$-sets which we denote $\GSet.$

For $x\in X$ the \emph{stabilizer} is defined
$$
    \Stab(x) := \{ g\in G \ | \ gx = x \}.
$$
This is clearly a subgroup of $G$.
Dually, 
the \emph{fixed point set} of $g\in G$
is the subset of $X$ given by
$$
    \Fix(g) := \{ x\in X \ | \ gx = x \}.
$$
\todo{In what sense are these really dual?}

The \emph{orbit} of $x\in X$ is the set
$$
    \Orbit(x) := \{ gx \ | \ g\in G \}.
$$

The following lemma shows that the stabilizers of
points in the same orbit are related by conjugation.
\begin{samepage}
\dolemma{\numitem{lemmaStabConj}}
Given a $G$-set $X$, with $x\in X, g\in G$, we have
$$
    \Stab(gx) = g\Stab(x) g^{-1}.
$$
\tombstone
\end{samepage}

We now state the \emph{orbit-stabilizer theorem}.
\dotheorem{\numitem{thmOS}}%{\bf (orbit-stabilizer)}
Given a $G$-set $X$ and a point $x\in X$
there is a bijection of sets:
$$
    \Orbit(x) \times \Stab(x) \cong G.
$$

\doproof
Let $H$ be the subgroup
$$
    H = \Stab(x).
$$
Then $G$ is partitioned into cosets $\{gH\ |\ g\in G\}.$
We claim that this set of cosets is in bijection with
the orbit of $x$.
The bijection is given by
\begin{align*}
    \Orbit(x) &\to \{gH \ |\ g\in G\} \\
    gx  &\mapsto  gH.
\end{align*}
To show that this is well defined,
let $gx=hx$ for $g,h\in G.$
Then $h^{-1}gx = x$ and so $h^{-1}g\in H,$
and the cosets $gH$ and $hH$ are identical.
\todo{finish}
\tombstone

% https://math.stackexchange.com/questions/1659075/category-theoretic-relation-between-orbit-stabilizer-and-rank-nullity-theorems/2168381
\doexample{\numitem{exvec}}
The rank-nullity theorem says that given
a linear map $A:V\to V$,
$$
    \Dim(\Image(A)) + \Dim(\Ker(A)) = \Dim(V).
$$
Such a map also gives a group action:
it is the additive group of $V$ acting on the set $V$
by addition. That is, any $v\in V$ acts on $x\in V$ as
$v: x \mapsto x+Av.$ 

Now we see that given any $x\in V$ the stabilizer subgroup
$\Stab(x)$ of this action is precisely the kernel
of $A.$ The orbit of $x$ is $x$ plus the image of $A.$

Working with a vector space over a finite field,
we can take the cardinality of these sets as in the formula
$|\Orbit(x)||\Stab(x)| = |G|$ and 
take the logarithm of this where the base is
the size of the field and we get exactly the rank-nullity
equation.

Over an infinite field this doesn't work and
we need to think more along the lines of a categorified
orbit-stabilizer theorem. 
\todo{Does this really work?}
In this case, for each $x\in
V$ we can find a bijection:

$$
\Orbit(x) \cong G / \Stab(x)
$$
and this bijection gives us the First Isomorphism Theorem:
$$
\Image(A) \cong V / \Ker(A). 
$$ 
\todo{Question: 
Can we bootstrap this one more time in order
to say something about the size of the homology 
groups in a chain complex?
Perhaps thinking of a (length 2) 
chain complex as a 2-category?}
\tombstone

For a $G$-set $X$, the \emph{orbit space} is defined as
the set of orbits:
$$
    G\backslash X := \{\Orbit(x)\}_{x\in X}.
$$
For a $G$-invariant map of sets $f:X\to Y$, 
ie., $f(x)=f(gx)$ for all $g\in G$,
we can find a unique map $\tilde{f}: G\backslash X\to Y$ such that
\[
\begin{tikzcd}
 X \arrow{rd}{p} \arrow{r}{f} &  Y \\
  &  G\backslash X \arrow{u}{\tilde{f}} 
\end{tikzcd}
\]
Where $p$ is the projection map $p:X\to G\backslash X.$

The following is known as ``Burnside's lemma''.
\dotheorem{\numitem{thmBurnside}}%{\bf (Burnside's lemma)}
Given a $G$-set $X$,
$$
    |G\backslash X| = \frac{1}{|G|} \sum_{g\in G} | \Fix(g) |.
$$
\doproof
\todo{todo}
\tombstone

An action is \emph{faithful} when for each $g\in G$ with
$g\ne 1$ we have $\Fix(g)\ne X.$
An action is \emph{free} when for each $g\in G$ with
$g\ne 1$ we have $\Fix(g)=\phi.$

\todo{define a torsor}

A $G$-set $X$ with only one orbit is called \emph{simple} 
(or \emph{transitive}, or \emph{indecomposable}).

Given two $G$-sets $X$ and $Y$, we define the sum $X+Y$ 
\todo{...} and the product $X\times Y$ \todo{...}

The following theorem shows that 
$G$-sets are semi-simple.
\dotheorem{\numitem{F1semisimple}}
For a $G$-set $X$, we have
$$
    X = \sum_{i=1}^{n} X_i
$$
with $X_i$ simple, and the summation is unique up to reordering.
\doproof
Easy.
\tombstone

The next theorem is Schur's lemma for $G$-sets.
\dotheorem{\numitem{F1schur}}
Given $G$-sets $X$ and $Y$:

(a) For $f:X\to Y,$ we have $f(X)$ is a $G$-set, $f(X)\subseteq Y.$

(b) For $f:X\to Y,$ and $X\ne\phi$,
if $Y$ is simple then $f$ is surjective.

(c) For $f:X\to X$ with $X$ simple, $f$ is an automorpism.

\doproof
Easy.
\tombstone


\subsection{The category of canonical orbits}

Taken from \cite{Bredon2006}.




\subsection{Hecke operators}

Let $\Complex[X]$ denote the complex vector space with basis $X.$
Evidently, when $X$ is a $G$-set, we get a $\Complex$-linear
representation of $G$ on $\Complex[X].$
A $\Complex$-linear representation of $G$ obtained in this
way we call a \emph{permutation representation}.

Given a point $x\in X$ we denote the corresponding basis
vector in $\Complex[X]$ as $\ket{x},$
and corresponding dual vector as $\bra{x}.$
We also denote generic vectors in $\Complex[X]$ by $\ket{v}$, $\ket{u}$, etc.

Given $G$-sets $X$ and $Y$, and points $x\in X, y\in Y$
we define the \emph{Hecke operator} as the linear operator
$$
    \Complex[X] \overset{r_{x,y}}{\longrightarrow} \Complex[Y]
$$
given by
$$
    r_{x,y} := \frac{1}{|\Stab(x)||\Stab(y)|} 
    \sum_{g\in G} \ket{gy}\bra{gx}.
$$

\dolemma{\numitem{lemmaHecke1}}
The Hecke operators are $G$-rep homomorphisms:
$$
    r_{x,y} \in \Hom_{\GRep}(\Complex[X], \Complex[Y]).
$$

\doproof
We need to show that 
$gr_{x,y}\ket{v} = r_{x,y} g\ket{v}$ for $\ket{v}\in\Complex[X], g\in G.$
By linearity we need only consider this equation on basis vectors:
$gr_{x,y}\ket{x'} = r_{x,y} g\ket{x'}$ for $x'\in X, g\in G.$
Computing:
\begin{align*}
    \mathrm{LHS} &= g\sum_{h\in H} \ket{hy} \\
     &= \sum_{h\in H} gh \ket{y} \\
    \mathrm{RHS} &= \sum_{h\in G} \ket{hy}\braket{hx}{gx} \\
     &= \sum_{h\in G, hx=gx} \ket{hy}\\
     &= \sum_{h\in H} gh \ket{y}.
\end{align*}
\todo{what is $H$?}
\tombstone


\dotheorem{\numitem{thm1}}
Given two permutation representation $\Complex[X]$
and $\Complex[Y]$ of a group $G$,
the Hecke operators
$$
    \{ r_{x,y} \ | \ x\in X, y\in Y \}
$$
form a basis for the linear space
$$
\Hom_{\GRep}(\Complex[X], \Complex[Y]).
$$

\doproof
Let $f\in \Hom_{\GRep}(\Complex[X], \Complex[Y]).$
Then for any $g\in G, x\in X, y\in Y$ we have:
\begin{align*}
    gf\ket{x} &= fg\ket{x} \\ 
    f\ket{x} &= g^{-1}fg\ket{x} \\ 
    \bra{y}f\ket{x} &= \bra{y}g^{-1}fg\ket{x} \\ 
              &= \bra{gy}f\ket{gx}.
\end{align*}
ie., the matrix for $f$ is constant on the orbits of
$X\times Y$ and so $f$ is a sum of Hecke operators.
\tombstone

\docorollary{\numitem{Hecke2}}
Given a doubly transitive action $G\to\Aut(X)$
the permutation representation $\Complex[X]$
breaks into exactly two irreducible representations. % in $\GRep.$
\doproof
There are two Hecke operators corresponding to the
diagonal matrix, and the off-diagonal matrix.
The result follows by the previous theorem and Schur's lemma.
\tombstone

\doproposition{\numitem{doublecosets}}
For a group $G$ with subgroups $H, K$,
the orbits of the $G$-set $G/H\times G/K$ 
are in bijection with the double cosets $\{HgK\}_{g\in G}.$
\doproof
Define the function 
\begin{align*}
f : G/H\times G/K &\to \{HgK\}_{g\in G} \\
    (aH, bK) &\mapsto Ha^{-1}bK.
\end{align*}
\todo{finish...}
\tombstone

\section{Examples}

\section{Bibliographic notes}

Group actions with applications to group theory: \cite{ConradGroup,ConradTransitive}.

See \cite{Dress1971}.


\bibliography{refs}{}
\bibliographystyle{abbrv}



\end{document}



